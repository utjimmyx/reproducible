\documentclass[]{article}
\usepackage{lmodern}
\usepackage{amssymb,amsmath}
\usepackage{ifxetex,ifluatex}
\usepackage{fixltx2e} % provides \textsubscript
\ifnum 0\ifxetex 1\fi\ifluatex 1\fi=0 % if pdftex
  \usepackage[T1]{fontenc}
  \usepackage[utf8]{inputenc}
\else % if luatex or xelatex
  \ifxetex
    \usepackage{mathspec}
  \else
    \usepackage{fontspec}
  \fi
  \defaultfontfeatures{Ligatures=TeX,Scale=MatchLowercase}
\fi
% use upquote if available, for straight quotes in verbatim environments
\IfFileExists{upquote.sty}{\usepackage{upquote}}{}
% use microtype if available
\IfFileExists{microtype.sty}{%
\usepackage{microtype}
\UseMicrotypeSet[protrusion]{basicmath} % disable protrusion for tt fonts
}{}
\usepackage[margin=1in]{geometry}
\usepackage{hyperref}
\hypersetup{unicode=true,
            pdftitle={Health and Economic Impact of Weather Events in the US},
            pdfauthor={zhenning xu},
            pdfborder={0 0 0},
            breaklinks=true}
\urlstyle{same}  % don't use monospace font for urls
\usepackage{color}
\usepackage{fancyvrb}
\newcommand{\VerbBar}{|}
\newcommand{\VERB}{\Verb[commandchars=\\\{\}]}
\DefineVerbatimEnvironment{Highlighting}{Verbatim}{commandchars=\\\{\}}
% Add ',fontsize=\small' for more characters per line
\usepackage{framed}
\definecolor{shadecolor}{RGB}{248,248,248}
\newenvironment{Shaded}{\begin{snugshade}}{\end{snugshade}}
\newcommand{\KeywordTok}[1]{\textcolor[rgb]{0.13,0.29,0.53}{\textbf{#1}}}
\newcommand{\DataTypeTok}[1]{\textcolor[rgb]{0.13,0.29,0.53}{#1}}
\newcommand{\DecValTok}[1]{\textcolor[rgb]{0.00,0.00,0.81}{#1}}
\newcommand{\BaseNTok}[1]{\textcolor[rgb]{0.00,0.00,0.81}{#1}}
\newcommand{\FloatTok}[1]{\textcolor[rgb]{0.00,0.00,0.81}{#1}}
\newcommand{\ConstantTok}[1]{\textcolor[rgb]{0.00,0.00,0.00}{#1}}
\newcommand{\CharTok}[1]{\textcolor[rgb]{0.31,0.60,0.02}{#1}}
\newcommand{\SpecialCharTok}[1]{\textcolor[rgb]{0.00,0.00,0.00}{#1}}
\newcommand{\StringTok}[1]{\textcolor[rgb]{0.31,0.60,0.02}{#1}}
\newcommand{\VerbatimStringTok}[1]{\textcolor[rgb]{0.31,0.60,0.02}{#1}}
\newcommand{\SpecialStringTok}[1]{\textcolor[rgb]{0.31,0.60,0.02}{#1}}
\newcommand{\ImportTok}[1]{#1}
\newcommand{\CommentTok}[1]{\textcolor[rgb]{0.56,0.35,0.01}{\textit{#1}}}
\newcommand{\DocumentationTok}[1]{\textcolor[rgb]{0.56,0.35,0.01}{\textbf{\textit{#1}}}}
\newcommand{\AnnotationTok}[1]{\textcolor[rgb]{0.56,0.35,0.01}{\textbf{\textit{#1}}}}
\newcommand{\CommentVarTok}[1]{\textcolor[rgb]{0.56,0.35,0.01}{\textbf{\textit{#1}}}}
\newcommand{\OtherTok}[1]{\textcolor[rgb]{0.56,0.35,0.01}{#1}}
\newcommand{\FunctionTok}[1]{\textcolor[rgb]{0.00,0.00,0.00}{#1}}
\newcommand{\VariableTok}[1]{\textcolor[rgb]{0.00,0.00,0.00}{#1}}
\newcommand{\ControlFlowTok}[1]{\textcolor[rgb]{0.13,0.29,0.53}{\textbf{#1}}}
\newcommand{\OperatorTok}[1]{\textcolor[rgb]{0.81,0.36,0.00}{\textbf{#1}}}
\newcommand{\BuiltInTok}[1]{#1}
\newcommand{\ExtensionTok}[1]{#1}
\newcommand{\PreprocessorTok}[1]{\textcolor[rgb]{0.56,0.35,0.01}{\textit{#1}}}
\newcommand{\AttributeTok}[1]{\textcolor[rgb]{0.77,0.63,0.00}{#1}}
\newcommand{\RegionMarkerTok}[1]{#1}
\newcommand{\InformationTok}[1]{\textcolor[rgb]{0.56,0.35,0.01}{\textbf{\textit{#1}}}}
\newcommand{\WarningTok}[1]{\textcolor[rgb]{0.56,0.35,0.01}{\textbf{\textit{#1}}}}
\newcommand{\AlertTok}[1]{\textcolor[rgb]{0.94,0.16,0.16}{#1}}
\newcommand{\ErrorTok}[1]{\textcolor[rgb]{0.64,0.00,0.00}{\textbf{#1}}}
\newcommand{\NormalTok}[1]{#1}
\usepackage{graphicx,grffile}
\makeatletter
\def\maxwidth{\ifdim\Gin@nat@width>\linewidth\linewidth\else\Gin@nat@width\fi}
\def\maxheight{\ifdim\Gin@nat@height>\textheight\textheight\else\Gin@nat@height\fi}
\makeatother
% Scale images if necessary, so that they will not overflow the page
% margins by default, and it is still possible to overwrite the defaults
% using explicit options in \includegraphics[width, height, ...]{}
\setkeys{Gin}{width=\maxwidth,height=\maxheight,keepaspectratio}
\IfFileExists{parskip.sty}{%
\usepackage{parskip}
}{% else
\setlength{\parindent}{0pt}
\setlength{\parskip}{6pt plus 2pt minus 1pt}
}
\setlength{\emergencystretch}{3em}  % prevent overfull lines
\providecommand{\tightlist}{%
  \setlength{\itemsep}{0pt}\setlength{\parskip}{0pt}}
\setcounter{secnumdepth}{0}
% Redefines (sub)paragraphs to behave more like sections
\ifx\paragraph\undefined\else
\let\oldparagraph\paragraph
\renewcommand{\paragraph}[1]{\oldparagraph{#1}\mbox{}}
\fi
\ifx\subparagraph\undefined\else
\let\oldsubparagraph\subparagraph
\renewcommand{\subparagraph}[1]{\oldsubparagraph{#1}\mbox{}}
\fi

%%% Use protect on footnotes to avoid problems with footnotes in titles
\let\rmarkdownfootnote\footnote%
\def\footnote{\protect\rmarkdownfootnote}

%%% Change title format to be more compact
\usepackage{titling}

% Create subtitle command for use in maketitle
\providecommand{\subtitle}[1]{
  \posttitle{
    \begin{center}\large#1\end{center}
    }
}

\setlength{\droptitle}{-2em}

  \title{Health and Economic Impact of Weather Events in the US}
    \pretitle{\vspace{\droptitle}\centering\huge}
  \posttitle{\par}
    \author{zhenning xu}
    \preauthor{\centering\large\emph}
  \postauthor{\par}
      \predate{\centering\large\emph}
  \postdate{\par}
    \date{May 18, 2019}


\begin{document}
\maketitle

{
\setcounter{tocdepth}{2}
\tableofcontents
}
\subsection{Synopsis}\label{synopsis}

The basic goal of this assignment is to explore the NOAA Storm Database
which contains events from 1950 to Nov 2011 and answer some basic
questions about severe weather events. Will try to analyze \& present my
findings in the following areas:

Across the United States, which types of events are most harmful with
respect to population health? Across the United States, which types of
events have the greatest economic consequences? Introduction Storms and
other severe weather events can cause both public health and economic
problems for communities and municipalities. Many severe events can
result in fatalities, injuries, and property damage, and preventing such
outcomes to the extent possible is a key concern.

This project involves exploring the U.S. National Oceanic and
Atmospheric Administration's (NOAA) storm Database. This Database tracks
characteristics of major storms and weather events in the United States,
including when and where they occur, as well as estimates of any
fatalities, injuries, and property damage.

\subsection{Data}\label{data}

The stormData for this assignment come in the form of a
comma-separated-value file compressed via the bzip2 algorithm to reduce
its size. You can download the file from the course web site:

Storm Data{[}47Mb{]}

There is also some documentation of the Database available. Here you
will find how some of the variables are constructed/defined.

National Weather Service Storm Data Documentation National Climatic
stormData Center Storm Events FAQ The events in the Database start in
the year 1950 and end in November 2011. In the earlier years of the
Database there are generally fewer events recorded, most likely due to a
lack of good records. More recent years should be considered more
complete.

\subsection{Data Processing}\label{data-processing}

The analysis was performed on
\href{http://www.ncdc.noaa.gov/stormevents/ftp.jsp}{Storm Events
Database}, provided by \href{http://www.ncdc.noaa.gov/}{National
Climatic Data Center}. The data is from a comma-separated-value file
available
\href{https://d396qusza40orc.cloudfront.net/repdata\%2Fdata\%2FStormData.csv.bz2}{here}.
There is also some documentation of the data available
\href{https://d396qusza40orc.cloudfront.net/repdata\%2Fpeer2_doc\%2Fpd01016005curr.pdf}{here}.

The first step is to read the data into a data frame.

\begin{Shaded}
\begin{Highlighting}[]
\KeywordTok{setwd}\NormalTok{(}\StringTok{"C:/Users/xzhenning/Documents/R/spring 2019/"}\NormalTok{)}
\NormalTok{storm <-}\StringTok{ }\KeywordTok{read.csv}\NormalTok{(}\StringTok{"C:./data/repdata_data_StormData.csv.bz2"}\NormalTok{)}
\end{Highlighting}
\end{Shaded}

Before the analysis, the data need some preprocessing. Event types don't
have a specific format. For instance, there are events with types
\texttt{Frost/Freeze}, \texttt{FROST/FREEZE} and
\texttt{FROST\textbackslash{}\textbackslash{}FREEZE} which obviously
refer to the same type of event.

\begin{Shaded}
\begin{Highlighting}[]
\CommentTok{# number of unique event types}
\KeywordTok{length}\NormalTok{(}\KeywordTok{unique}\NormalTok{(storm}\OperatorTok{$}\NormalTok{EVTYPE))}
\end{Highlighting}
\end{Shaded}

\begin{verbatim}
## [1] 985
\end{verbatim}

\begin{Shaded}
\begin{Highlighting}[]
\CommentTok{# translate all letters to lowercase}
\NormalTok{event_types <-}\StringTok{ }\KeywordTok{tolower}\NormalTok{(storm}\OperatorTok{$}\NormalTok{EVTYPE)}
\CommentTok{# replace all punct. characters with a space}
\NormalTok{event_types <-}\StringTok{ }\KeywordTok{gsub}\NormalTok{(}\StringTok{"[[:blank:][:punct:]+]"}\NormalTok{, }\StringTok{" "}\NormalTok{, event_types)}
\KeywordTok{length}\NormalTok{(}\KeywordTok{unique}\NormalTok{(event_types))}
\end{Highlighting}
\end{Shaded}

\begin{verbatim}
## [1] 874
\end{verbatim}

\begin{Shaded}
\begin{Highlighting}[]
\CommentTok{# update the data frame}
\NormalTok{storm}\OperatorTok{$}\NormalTok{EVTYPE <-}\StringTok{ }\NormalTok{event_types}
\end{Highlighting}
\end{Shaded}

No further data preprocessing was performed although the event type
field can be processed further to merge event types such as
\texttt{tstm\ wind} and \texttt{thunderstorm\ wind}.

After performing data cleaning, as expected, the number of unique event
types reduce significantly. For further analysis, the cleaned event
types are used.

\subsection{Dangerous Events with respect to Population
Health}\label{dangerous-events-with-respect-to-population-health}

To find the event types that are most harmful to population health, the
number of casualties are aggregated by the event type.

\begin{Shaded}
\begin{Highlighting}[]
\KeywordTok{library}\NormalTok{(plyr)}
\NormalTok{casualties <-}\StringTok{ }\KeywordTok{ddply}\NormalTok{(storm, .(EVTYPE), summarize,}
                    \DataTypeTok{fatalities =} \KeywordTok{sum}\NormalTok{(FATALITIES),}
                    \DataTypeTok{injuries =} \KeywordTok{sum}\NormalTok{(INJURIES))}
\CommentTok{# Find events that caused most death and injury}
\NormalTok{fatal_events <-}\StringTok{ }\KeywordTok{head}\NormalTok{(casualties[}\KeywordTok{order}\NormalTok{(casualties}\OperatorTok{$}\NormalTok{fatalities, }\DataTypeTok{decreasing =}\NormalTok{ T), ], }\DecValTok{10}\NormalTok{)}
\NormalTok{injury_events <-}\StringTok{ }\KeywordTok{head}\NormalTok{(casualties[}\KeywordTok{order}\NormalTok{(casualties}\OperatorTok{$}\NormalTok{injuries, }\DataTypeTok{decreasing =}\NormalTok{ T), ], }\DecValTok{10}\NormalTok{)}
\end{Highlighting}
\end{Shaded}

Top 10 events that caused largest number of deaths are

\begin{Shaded}
\begin{Highlighting}[]
\NormalTok{fatal_events[, }\KeywordTok{c}\NormalTok{(}\StringTok{"EVTYPE"}\NormalTok{, }\StringTok{"fatalities"}\NormalTok{)]}
\end{Highlighting}
\end{Shaded}

\begin{verbatim}
##             EVTYPE fatalities
## 741        tornado       5633
## 116 excessive heat       1903
## 138    flash flood        978
## 240           heat        937
## 410      lightning        816
## 762      tstm wind        504
## 154          flood        470
## 515    rip current        368
## 314      high wind        248
## 19       avalanche        224
\end{verbatim}

Top 10 events that caused most number of injuries are

\begin{Shaded}
\begin{Highlighting}[]
\NormalTok{injury_events[, }\KeywordTok{c}\NormalTok{(}\StringTok{"EVTYPE"}\NormalTok{, }\StringTok{"injuries"}\NormalTok{)]}
\end{Highlighting}
\end{Shaded}

\begin{verbatim}
##                EVTYPE injuries
## 741           tornado    91346
## 762         tstm wind     6957
## 154             flood     6789
## 116    excessive heat     6525
## 410         lightning     5230
## 240              heat     2100
## 382         ice storm     1975
## 138       flash flood     1777
## 671 thunderstorm wind     1488
## 209              hail     1361
\end{verbatim}

\section{\#\# Economic Effects of Weather
Events}\label{economic-effects-of-weather-events}

To analyze the impact of weather events on the economy, available
property damage and crop damage reportings/estimates were used.

Note: in the raw data, the property damage is represented with two
fields, a number \texttt{PROPDMG} in dollars and the exponent
\texttt{PROPDMGEXP}. Similarly, the crop damage is represented using two
fields, \texttt{CROPDMG} and \texttt{CROPDMGEXP}. The first step in the
analysis is to calculate the property and crop damage for each event.

\begin{Shaded}
\begin{Highlighting}[]
\NormalTok{exp_transform <-}\StringTok{ }\ControlFlowTok{function}\NormalTok{(e) \{}
    \CommentTok{# h -> hundred, k -> thousand, m -> million, b -> billion}
    \ControlFlowTok{if}\NormalTok{ (e }\OperatorTok\StringTok{ }\KeywordTok{c}\NormalTok{(}\StringTok{'h'}\NormalTok{, }\StringTok{'H'}\NormalTok{))}
        \KeywordTok{return}\NormalTok{(}\DecValTok{2}\NormalTok{)}
    \ControlFlowTok{else} \ControlFlowTok{if}\NormalTok{ (e }\OperatorTok\StringTok{ }\KeywordTok{c}\NormalTok{(}\StringTok{'k'}\NormalTok{, }\StringTok{'K'}\NormalTok{))}
        \KeywordTok{return}\NormalTok{(}\DecValTok{3}\NormalTok{)}
    \ControlFlowTok{else} \ControlFlowTok{if}\NormalTok{ (e }\OperatorTok\StringTok{ }\KeywordTok{c}\NormalTok{(}\StringTok{'m'}\NormalTok{, }\StringTok{'M'}\NormalTok{))}
        \KeywordTok{return}\NormalTok{(}\DecValTok{6}\NormalTok{)}
    \ControlFlowTok{else} \ControlFlowTok{if}\NormalTok{ (e }\OperatorTok\StringTok{ }\KeywordTok{c}\NormalTok{(}\StringTok{'b'}\NormalTok{, }\StringTok{'B'}\NormalTok{))}
        \KeywordTok{return}\NormalTok{(}\DecValTok{9}\NormalTok{)}
    \ControlFlowTok{else} \ControlFlowTok{if}\NormalTok{ (}\OperatorTok{!}\KeywordTok{is.na}\NormalTok{(}\KeywordTok{as.numeric}\NormalTok{(e))) }\CommentTok{# if a digit}
        \KeywordTok{return}\NormalTok{(}\KeywordTok{as.numeric}\NormalTok{(e))}
    \ControlFlowTok{else} \ControlFlowTok{if}\NormalTok{ (e }\OperatorTok\StringTok{ }\KeywordTok{c}\NormalTok{(}\StringTok{''}\NormalTok{, }\StringTok{'-'}\NormalTok{, }\StringTok{'?'}\NormalTok{, }\StringTok{'+'}\NormalTok{))}
        \KeywordTok{return}\NormalTok{(}\DecValTok{0}\NormalTok{)}
    \ControlFlowTok{else}\NormalTok{ \{}
        \KeywordTok{stop}\NormalTok{(}\StringTok{"Invalid exponent value."}\NormalTok{)}
\NormalTok{    \}}
\NormalTok{\}}
\end{Highlighting}
\end{Shaded}

\begin{Shaded}
\begin{Highlighting}[]
\NormalTok{prop_dmg_exp <-}\StringTok{ }\KeywordTok{sapply}\NormalTok{(storm}\OperatorTok{$}\NormalTok{PROPDMGEXP, }\DataTypeTok{FUN=}\NormalTok{exp_transform)}
\NormalTok{storm}\OperatorTok{$}\NormalTok{prop_dmg <-}\StringTok{ }\NormalTok{storm}\OperatorTok{$}\NormalTok{PROPDMG }\OperatorTok{*}\StringTok{ }\NormalTok{(}\DecValTok{10} \OperatorTok{**}\StringTok{ }\NormalTok{prop_dmg_exp)}
\NormalTok{crop_dmg_exp <-}\StringTok{ }\KeywordTok{sapply}\NormalTok{(storm}\OperatorTok{$}\NormalTok{CROPDMGEXP, }\DataTypeTok{FUN=}\NormalTok{exp_transform)}
\NormalTok{storm}\OperatorTok{$}\NormalTok{crop_dmg <-}\StringTok{ }\NormalTok{storm}\OperatorTok{$}\NormalTok{CROPDMG }\OperatorTok{*}\StringTok{ }\NormalTok{(}\DecValTok{10} \OperatorTok{**}\StringTok{ }\NormalTok{crop_dmg_exp)}
\end{Highlighting}
\end{Shaded}

\begin{Shaded}
\begin{Highlighting}[]
\CommentTok{# Compute the economic loss by event type}
\KeywordTok{library}\NormalTok{(plyr)}
\NormalTok{econ_loss <-}\StringTok{ }\KeywordTok{ddply}\NormalTok{(storm, .(EVTYPE), summarize,}
                   \DataTypeTok{prop_dmg =} \KeywordTok{sum}\NormalTok{(prop_dmg),}
                   \DataTypeTok{crop_dmg =} \KeywordTok{sum}\NormalTok{(crop_dmg))}
\CommentTok{# filter out events that caused no economic loss}
\NormalTok{econ_loss <-}\StringTok{ }\NormalTok{econ_loss[(econ_loss}\OperatorTok{$}\NormalTok{prop_dmg }\OperatorTok{>}\StringTok{ }\DecValTok{0} \OperatorTok{|}\StringTok{ }\NormalTok{econ_loss}\OperatorTok{$}\NormalTok{crop_dmg }\OperatorTok{>}\StringTok{ }\DecValTok{0}\NormalTok{), ]}
\NormalTok{prop_dmg_events <-}\StringTok{ }\KeywordTok{head}\NormalTok{(econ_loss[}\KeywordTok{order}\NormalTok{(econ_loss}\OperatorTok{$}\NormalTok{prop_dmg, }\DataTypeTok{decreasing =}\NormalTok{ T), ], }\DecValTok{10}\NormalTok{)}
\NormalTok{crop_dmg_events <-}\StringTok{ }\KeywordTok{head}\NormalTok{(econ_loss[}\KeywordTok{order}\NormalTok{(econ_loss}\OperatorTok{$}\NormalTok{crop_dmg, }\DataTypeTok{decreasing =}\NormalTok{ T), ], }\DecValTok{10}\NormalTok{)}
\end{Highlighting}
\end{Shaded}

Top 10 events that caused most property damage (in dollars) are as
follows

\begin{Shaded}
\begin{Highlighting}[]
\NormalTok{prop_dmg_events[, }\KeywordTok{c}\NormalTok{(}\StringTok{"EVTYPE"}\NormalTok{, }\StringTok{"prop_dmg"}\NormalTok{)]}
\end{Highlighting}
\end{Shaded}

\begin{verbatim}
##                 EVTYPE     prop_dmg
## 138        flash flood 6.820237e+13
## 697 thunderstorm winds 2.086532e+13
## 741            tornado 1.078951e+12
## 209               hail 3.157558e+11
## 410          lightning 1.729433e+11
## 154              flood 1.446577e+11
## 366  hurricane typhoon 6.930584e+10
## 166           flooding 5.920826e+10
## 585        storm surge 4.332354e+10
## 270         heavy snow 1.793259e+10
\end{verbatim}

Similarly, the events that caused biggest crop damage are

\begin{Shaded}
\begin{Highlighting}[]
\NormalTok{crop_dmg_events[, }\KeywordTok{c}\NormalTok{(}\StringTok{"EVTYPE"}\NormalTok{, }\StringTok{"crop_dmg"}\NormalTok{)]}
\end{Highlighting}
\end{Shaded}

\begin{verbatim}
##                EVTYPE    crop_dmg
## 84            drought 13972566000
## 154             flood  5661968450
## 519       river flood  5029459000
## 382         ice storm  5022113500
## 209              hail  3025974480
## 357         hurricane  2741910000
## 366 hurricane typhoon  2607872800
## 138       flash flood  1421317100
## 125      extreme cold  1312973000
## 185      frost freeze  1094186000
\end{verbatim}

\subsection{Plots}\label{plots}

The following plots show top dangerous weather event types.

\begin{Shaded}
\begin{Highlighting}[]
\KeywordTok{library}\NormalTok{(ggplot2)}
\end{Highlighting}
\end{Shaded}

\begin{verbatim}
## Registered S3 methods overwritten by 'ggplot2':
##   method         from 
##   [.quosures     rlang
##   c.quosures     rlang
##   print.quosures rlang
\end{verbatim}

\begin{Shaded}
\begin{Highlighting}[]
\KeywordTok{library}\NormalTok{(gridExtra)}
\CommentTok{# Set the levels in order}
\NormalTok{p1 <-}\StringTok{ }\KeywordTok{ggplot}\NormalTok{(}\DataTypeTok{data=}\NormalTok{fatal_events,}
             \KeywordTok{aes}\NormalTok{(}\DataTypeTok{x=}\KeywordTok{reorder}\NormalTok{(EVTYPE, fatalities), }\DataTypeTok{y=}\NormalTok{fatalities, }\DataTypeTok{fill=}\NormalTok{fatalities)) }\OperatorTok{+}
\StringTok{    }\KeywordTok{geom_bar}\NormalTok{(}\DataTypeTok{stat=}\StringTok{"identity"}\NormalTok{) }\OperatorTok{+}
\StringTok{    }\KeywordTok{coord_flip}\NormalTok{() }\OperatorTok{+}
\StringTok{    }\KeywordTok{ylab}\NormalTok{(}\StringTok{"Total number of fatalities"}\NormalTok{) }\OperatorTok{+}
\StringTok{    }\KeywordTok{xlab}\NormalTok{(}\StringTok{"Event type"}\NormalTok{) }\OperatorTok{+}
\StringTok{ }\KeywordTok{ggtitle}\NormalTok{ (}\StringTok{"Top 20 events for total fatalities"}\NormalTok{) }\OperatorTok{+}
\StringTok{    }\KeywordTok{theme}\NormalTok{(}\DataTypeTok{axis.text.x=}\KeywordTok{element_text}\NormalTok{(}\DataTypeTok{angle=}\DecValTok{45}\NormalTok{,}\DataTypeTok{hjust=}\DecValTok{1}\NormalTok{))}
\NormalTok{p2 <-}\StringTok{ }\KeywordTok{ggplot}\NormalTok{(}\DataTypeTok{data=}\NormalTok{injury_events,}
             \KeywordTok{aes}\NormalTok{(}\DataTypeTok{x=}\KeywordTok{reorder}\NormalTok{(EVTYPE, injuries), }\DataTypeTok{y=}\NormalTok{injuries, }\DataTypeTok{fill=}\NormalTok{injuries)) }\OperatorTok{+}
\StringTok{    }\KeywordTok{geom_bar}\NormalTok{(}\DataTypeTok{stat=}\StringTok{"identity"}\NormalTok{) }\OperatorTok{+}
\StringTok{    }\KeywordTok{coord_flip}\NormalTok{() }\OperatorTok{+}\StringTok{ }
\StringTok{    }\KeywordTok{ylab}\NormalTok{(}\StringTok{"Total number of injuries"}\NormalTok{) }\OperatorTok{+}
\StringTok{    }\KeywordTok{xlab}\NormalTok{(}\StringTok{"Event type"}\NormalTok{) }\OperatorTok{+}
\StringTok{ }\KeywordTok{ggtitle}\NormalTok{ (}\StringTok{"Top 20 events for total injuries"}\NormalTok{) }\OperatorTok{+}
\StringTok{    }\KeywordTok{theme}\NormalTok{(}\DataTypeTok{axis.text.x=}\KeywordTok{element_text}\NormalTok{(}\DataTypeTok{angle=}\DecValTok{45}\NormalTok{,}\DataTypeTok{hjust=}\DecValTok{1}\NormalTok{))}
\NormalTok{p1}
\end{Highlighting}
\end{Shaded}

\includegraphics{NOAA-storm_files/figure-latex/unnamed-chunk-11-1.pdf}

\begin{Shaded}
\begin{Highlighting}[]
\NormalTok{p2}
\end{Highlighting}
\end{Shaded}

\includegraphics{NOAA-storm_files/figure-latex/unnamed-chunk-11-2.pdf}

Tornadoes cause most number of deaths and injuries among all event
types. There are more than 5,000 deaths and more than 10,000 injuries in
the last 60 years in US, due to tornadoes. The other event types that
are most dangerous with respect to population health are excessive heat
and flash floods.

\subsection{Economic impact of weather
events}\label{economic-impact-of-weather-events}

The following plot shows the most severe weather event types with
respect to economic cost that they have costed since 1950s.

\begin{Shaded}
\begin{Highlighting}[]
\KeywordTok{library}\NormalTok{(ggplot2)}
\KeywordTok{library}\NormalTok{(gridExtra)}
\CommentTok{# Set the levels in order}
\NormalTok{p1 <-}\StringTok{ }\KeywordTok{ggplot}\NormalTok{(}\DataTypeTok{data=}\NormalTok{prop_dmg_events,}
             \KeywordTok{aes}\NormalTok{(}\DataTypeTok{x=}\KeywordTok{reorder}\NormalTok{(EVTYPE, prop_dmg), }\DataTypeTok{y=}\KeywordTok{log10}\NormalTok{(prop_dmg), }\DataTypeTok{fill=}\NormalTok{prop_dmg )) }\OperatorTok{+}
\StringTok{    }\KeywordTok{geom_bar}\NormalTok{(}\DataTypeTok{stat=}\StringTok{"identity"}\NormalTok{) }\OperatorTok{+}
\StringTok{    }\KeywordTok{coord_flip}\NormalTok{() }\OperatorTok{+}
\StringTok{    }\KeywordTok{xlab}\NormalTok{(}\StringTok{"Event type"}\NormalTok{) }\OperatorTok{+}
\StringTok{    }\KeywordTok{ylab}\NormalTok{(}\StringTok{"Property damage in dollars (log-scale)"}\NormalTok{) }\OperatorTok{+}
\StringTok{   }\KeywordTok{ggtitle}\NormalTok{ (}\StringTok{"Top 20 events for total property damage"}\NormalTok{) }\OperatorTok{+}
\StringTok{    }\KeywordTok{theme}\NormalTok{(}\DataTypeTok{axis.text.x=}\KeywordTok{element_text}\NormalTok{(}\DataTypeTok{angle=}\DecValTok{45}\NormalTok{,}\DataTypeTok{hjust=}\DecValTok{1}\NormalTok{))}
\NormalTok{p2 <-}\StringTok{ }\KeywordTok{ggplot}\NormalTok{(}\DataTypeTok{data=}\NormalTok{crop_dmg_events,}
             \KeywordTok{aes}\NormalTok{(}\DataTypeTok{x=}\KeywordTok{reorder}\NormalTok{(EVTYPE, crop_dmg), }\DataTypeTok{y=}\NormalTok{crop_dmg, }\DataTypeTok{fill=}\NormalTok{crop_dmg)) }\OperatorTok{+}
\StringTok{    }\KeywordTok{geom_bar}\NormalTok{(}\DataTypeTok{stat=}\StringTok{"identity"}\NormalTok{) }\OperatorTok{+}
\StringTok{    }\KeywordTok{coord_flip}\NormalTok{() }\OperatorTok{+}\StringTok{ }
\StringTok{    }\KeywordTok{xlab}\NormalTok{(}\StringTok{"Event type"}\NormalTok{) }\OperatorTok{+}
\StringTok{    }\KeywordTok{ylab}\NormalTok{(}\StringTok{"Crop damage in dollars"}\NormalTok{) }\OperatorTok{+}\StringTok{ }
\StringTok{  }\KeywordTok{ggtitle}\NormalTok{ (}\StringTok{"Top 20 events for total crop damage"}\NormalTok{) }\OperatorTok{+}
\StringTok{    }\KeywordTok{theme}\NormalTok{(}\DataTypeTok{axis.text.x=}\KeywordTok{element_text}\NormalTok{(}\DataTypeTok{angle=}\DecValTok{45}\NormalTok{,}\DataTypeTok{hjust=}\DecValTok{1}\NormalTok{))}
\NormalTok{p1}
\end{Highlighting}
\end{Shaded}

\includegraphics{NOAA-storm_files/figure-latex/unnamed-chunk-12-1.pdf}

\begin{Shaded}
\begin{Highlighting}[]
\NormalTok{p2}
\end{Highlighting}
\end{Shaded}

\includegraphics{NOAA-storm_files/figure-latex/unnamed-chunk-12-2.pdf}

\subsection{conlcusions}\label{conlcusions}

The data shows that flash floods and thunderstorm winds cost the largest
property damages among weather-related natural diseasters.

The most severe weather event in terms of crop damage is drought. In the
last 50 years, drought has caused more than 10 billion dollars in loss.
Other severe crop-damage-causing event types include floods and hails.

Note:Property damages are converted to logarithmic scales due to large
range of values.


\end{document}
